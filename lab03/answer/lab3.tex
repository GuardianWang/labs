%%%%%%%%%%%%%%%%%%%%%%%%%%%%%%%%%%%%%%%%%%%%%%%%%%%%%%%%%%%%%%%%%%%%%
%
% CSCI 1230 Algo Answer Template
% This template is based on Brown CS course CSCI 1420 http://cs.brown.edu/courses/csci1420/resources.html
% and CSCI 1430 https://browncsci1430.github.io/webpage/.
%
%%%%%%%%%%%%%%%%%%%%%%%%%%%%%%%%%%%%%%%%%%%%%%%%%%%%%%%%%%%%%%%%%%%%%

\documentclass[12pt,letterpaper,onecolumn]{article}
\usepackage{fullpage}
\usepackage[top=2cm, bottom=4.5cm, left=2.5cm, right=2.5cm]{geometry}
\usepackage{amsmath,amsthm,amsfonts,amssymb,amscd}
\usepackage{multicol}
\usepackage{lastpage}
\usepackage{enumerate}
\usepackage{fancyhdr}
\usepackage{mathrsfs}
\usepackage{graphicx}
\usepackage{booktabs}
\usepackage{pythonhighlight}
\usepackage{xcolor}
\usepackage{listings}

\setlength{\parindent}{0.0in}
\setlength{\parskip}{0.05in}

\colorlet{mygray}{black!30}
\colorlet{mygreen}{green!60!blue}
\colorlet{mymauve}{red!60!blue}

% https://tex.stackexchange.com/questions/409705/c-code-change-the-font
\lstset{
  backgroundcolor=\color{gray!10},  
  basicstyle=\ttfamily,
  columns=fullflexible,
  breakatwhitespace=false,      
  breaklines=true,                
  captionpos=b,                    
  commentstyle=\color{mygreen}, 
  extendedchars=true,              
  frame=single,                   
  keepspaces=true,             
  keywordstyle=\color{blue},      
  language=c++,                 
  numbers=none,                
  numbersep=5pt,                   
  numberstyle=\tiny\color{blue}, 
  rulecolor=\color{mygray},        
  showspaces=false,               
  showtabs=false,                 
  stepnumber=5,                  
  stringstyle=\color{mymauve},    
  tabsize=3,                      
  title=\lstname                
}

% Edit these as appropriate
\newcommand\course{CSCI 1230 \\ Intro. to CG}
\newcommand\algono{03}
\newcommand\algoname{Transformations \& Scene Graphs}
\newcommand\semester{Fall 2021}            % <-- current semester

\pagestyle{fancyplain}
\headheight 35pt
\lhead{\course}
\chead{\Large{Lab \algono\ \algoname}}
\rhead{\semester \\ \today}
\lfoot{}
\cfoot{}
\rfoot{\small\thepage}
\headsep 1.5em

\begin{document}
\section{}
$C^\prime = RTSC$

\section{}
$M_1 M_2$

\section{}
\begin{multicols}{2} 
blue square: $M_1 M_3 M_6$

blue triangle: $M_1 M_3 M_7$

white circle: $M_1 M_4 M_8 M_{11}$

black square: $M_1 M_4 M_9 M_{12}$

blue rectangle: $M_1 M_4 M_{10}$
\end{multicols}

\section{}

We can save in each node the corresponding transformation
matrix.
To do this, we can apply DFS and the complexity would be 
$O(n)$ where $n$ is the number of nodes.

\begin{lstlisting}
class Node {
private:
    mat4 transformMat;
    int numChildren;
    Node[] children;
    mat4[] edgeTransforms;
    // other attributes
};
void dfs(Node node, mat4 nodeMat, mat4 edgeMat) {
    node.transformMat = nodeMat * edgeMat

    if (0 == node.numChildren) render(node);  // leaf
        
    for (int i = 0; i < node.numChildren; i++) {
        dfs(node.children[i], node.transformMat, node.edgeTransforms[i]);
    }
}
    

\end{lstlisting}

\end{document}
